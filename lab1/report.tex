\documentclass[a4paper,14pt]{article} % формат документа

\usepackage{amsmath}
\usepackage{cmap} % поиск в ПДФ
\usepackage[T2A]{fontenc} % кодировка
\usepackage[utf8]{inputenc} % кодировка исходного текста
\usepackage[english,russian]{babel} % локализация и переносы
\usepackage[left = 2cm, right = 1cm, top = 2cm, bottom = 2 cm]{geometry} % поля
\usepackage{listings}
\usepackage{graphicx} % для вставки рисунков
\usepackage{amsmath}
\graphicspath{{pictures/}}
\DeclareGraphicsExtensions{.pdf,.png,.jpg}
\newcommand{\anonsection}[1]{\section*{#1}\addcontentsline{toc}{section}{#1}}

\lstset{ %
	language=Python,                % Язык программирования 
	numbers=left,                   % С какой стороны нумеровать          
	frame=single,                    % Добавить рамку
}

\begin{document}

\textbf{Цель работы: } Изучить методы решения задачи Коши для обыкновенных дифференциальных уравнений, применив приближенно-аналитический метод Пикара и численный метод Эйлера в явном и неявном виде. 

Ищем решение уравнения:
$$
\left\{
  \begin{array}{ccc}
     u'(x)=x^2+u^2 \\
     u(0)=0 \\
  \end{array}
\right.
$$

\begin{enumerate}
\item \textbf{Метод Пикара}

$$y^{(1)}=\frac{x^3}{3}
~~~~~~~y^{(2)}=\frac{x^3}{3} + \frac{x^7}{63}
~~~~~~~y^{(3)}=\frac{x^3}{3}+ \frac{x^7}{63} + \frac{2x^{11}}{2079} + \frac{x^{15}}{59535}$$

$$y^{(4)}=\frac{x^3}{3}+ \frac{x^7}{63} + \frac{2x^{11}}{2079} + \frac{13x^{15}}{218295} + \frac{82x^{19}}{37328445} +  \frac{662x^{23}}{10438212015}+\frac{4x^{27}}{3341878155}+\frac{x^{31}}{109876902975}$$

\item \textbf{Метод Эйлера}
\begin{enumerate}
\item \textbf{Явный вид}

$$y_{n+1}=y_n+h\cdot f(x_n,y_n)\text{ , где }f(x,y)=x^2+y^2$$

\item \textbf{Неявный вид}

$$y_{n+1}=y_n+h\cdot f(x_{n+1},y_{n+1})\text{ , где }f(x,y)=x^2+y^2$$

\end{enumerate}
\end{enumerate}

 
\begin{lstlisting}[caption=Метод Пикара]
def pikar(approx, x):
    switcher = {
            1 : pow(x, 3) / 3.0,
            2 : pow(x, 3) / 3.0 + pow(x, 7) / 63.0,
            3 : pow(x, 3) / 3.0 + pow(x, 7) / 63.0 + 2 * pow(x, 11) / 2079.0
                + pow(x, 15) / 59535.0,
            4 : pow(x, 3) / 3.0 + pow(x, 7) / 63.0 + 2 * pow(x, 11) / 2079.0
                + 13 * pow(x, 15) / 218295.0 + 82 * pow(x, 19) / 37328445.0
                + 662 * pow(x, 23) / 10438212015.0 + 4 * pow(x, 27) / 3341878155.0
                + pow(x, 31) / 109876902975.0
            }
    return switcher.get(approx, "Invalid approx")
\end{lstlisting}
        
\begin{lstlisting}[caption=Метод Эйлера(явный)]
def explicit_function(x, h):
    f = 0
    x0 = h
    while (x0 < x + h / 2):
        f += h * (x0 * x0 + f * f)
        x0 += h
    return f
            }
    return switcher.get(approx, "Invalid approx")
\end{lstlisting}

\begin{lstlisting}[caption=Метод Эйлера(неявный)]
def notexplicit_function(x, h):
    f = 0
    x0 = h
    while (x0 < x + h):
        descr = 1 - 4 * h * (h * x0 * x0 + f)
        if (descr >= 0):
            f1 = (1 + sqrt(descr)) / 2 / h
            f2 = (1 - sqrt(descr)) / 2 / h
            f = f1 if f2 < 0 else f2 if f1 < 0 else min(f1, f2)
        x0 += h
    return f
\end{lstlisting}

\end{document}